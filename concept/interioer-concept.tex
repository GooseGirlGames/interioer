\documentclass[a4paper]{article}

% symlink to ~/.texinit/default.tex:
%   ln -s /home/jakob/uni/template.tex /home/jakob/.texinit/default.tex

%-----------------------------------------------------------------------
% LaTeX packages
%-----------------------------------------------------------------------

\usepackage[mono=false]{libertine}

\usepackage[english]{babel}
\selectlanguage{english}

\usepackage{textcomp}
\usepackage{amsfonts}
\usepackage{amsmath}
\usepackage{amsthm}
\usepackage{amssymb}
\usepackage{booktabs}
\usepackage[hmargin=1in,vmargin=1in]{geometry}
\usepackage[pdfborderstyle={/S/U/W 1}]{hyperref}
\usepackage{xspace}
\usepackage{lipsum}
\usepackage{graphicx}
\usepackage{float}
\usepackage{subfig}
%\usepackage[utf8]{inputenc}

\usepackage[symbol]{footmisc}
%\usepackage{indentfirst}
%\setlength\parindent{0pt} % disable indenting

% German bibliography
\usepackage{bibgerm}
\usepackage[square,numbers]{natbib}

\usepackage{array} % allows for things like `>{\bf}c` in tables

\usepackage{fancyhdr}

\usepackage{tablefootnote}


\pagestyle{fancy}
\fancyhf{}
\lhead{\textsc{Interiör: Concept}}
\rhead{\textsc{Goose Girl Games}}
%\rhead{\textsc{Erickson, Ruckel, Trojan}}
%\cfoot{--~\thepage~--}
\fancyfoot[C]{--~\thepage~--}

\fancypagestyle{plain}{%
  \fancyhf{}
  \renewcommand*{\headrulewidth}{0pt}
  \fancyfoot[C]{--~\thepage~--}
}

\usepackage{titling}

% Footnotes w/ symbols, not numbers
%\renewcommand{\thefootnote}{\fnsymbol{footnote}}

% Fix `Package Fancyhdr Warning: \headheight is too small (12.0pt): Make it at least 12.37082pt`
%\setlength{\headheight}{12.38pt}

%-----------------------------------------------------------------------
% Theorems
%-----------------------------------------------------------------------

\newtheorem{thm}{Theorem}[section]
\newtheorem{cor}[thm]{Corollary}
\newtheorem{lem}[thm]{Lemma}
\newtheorem{prop}[thm]{Proposition}
\theoremstyle{definition}
\newtheorem{defn}[thm]{Definition}
\theoremstyle{remark}
\newtheorem{rem}[thm]{Remark}

%-----------------------------------------------------------------------
% Macros
%-----------------------------------------------------------------------

\newcommand{\pmat}[1]{\begin{pmatrix} #1 \end{pmatrix}}

%-----------------------------------------------------------------------

\begin{document}

%-----------------------------------------------------------------------
% Title
%-----------------------------------------------------------------------
\author{Goose Girl Games}
\date{August 2021}

\title{\textbf{%
    Interiör: Concept
}}

\maketitle

\tableofcontents

\newpage

%\author{Fionn Edward Erickson, Jakob Béla Ruckel und Marie Trojan\\
%  \normalsize Matrikelnummern: 121588, 120858 und 121067\\


%-----------------------------------------------------------------------
% Content
%-----------------------------------------------------------------------

\newcommand{\ikea}{\textsc{Eeek}\xspace}
\newcommand{\smaland}{\textsc{Småland}\xspace}

\section{Premise}

You and your parents go on a typical Saturday \ikea trip,
but soon, your greatest fear becomes reality:
Your parents leave you behind at \smaland.

You're trying to leave the store, but to do so you must fight your way
through several layers of \ikea showrooms and, finally, the warehouse.

\section{Layers/Showrooms}

Each layer is a single showroom floor with one theme.
Layout and enemy spawns are randomly generated.

\subsection{Elements within a layer}

\subsubsection{Checkpoint}

There's multiple checkpoints with
pencils, measuring tape and a layout map.

\subsubsection{Cafeteria}

At the end of each layer there's an \ikea cafeteria,
at which the player can upgrade weapons and buy food.

\subsubsection{Family Section}

Requires \ikea Family Membership Card to enter.
Has exclusive items, so it's basically a treasure room.

\subsubsection{Boss}

One Boss per layer, chosen at random out of a set of bosses with the
layer's theme.

\subsection{Types of Layers}

\begin{itemize}
  \item Intro/Tutorial: \smaland
  \item Special (at half point?): As-Is Section (Fundgrube).
    Contains broken zombie versions of previously defeated pieces of
    furniture.
  \item Special: Full apartment ($\dots m^2$) showroom.
    Contains all (previously seen) types of furniture.
  \item Bathroom.
    Theme: Water and Electricity.
  \item Kitchen.
    Cutlery.
  \item Bedroom
  \item Children's Room
  \item Garden/Plants
  \item Living Room
  \item Carpets, Blankets, etc.
  \item Office
  \item Lighting
  \item End Boss: Warehouse
\end{itemize}

\subsection{Level Generation}

At the core, a level is a small kind of maze.
Ideas and requirements:
\begin{itemize}
  \item Entrance and exit (the latter with the cafeteria).
  \item Combinations of an item and an enemy (or group of enemies) that
    is best defeated with that item.
  \item Shortcuts:  Can be taken, but player will have to fight enemies
    with non-optimal weapons.
\end{itemize}

\section{Movement}

Dash/Dodge:
Player kind of stumbles after a dash, meaning there is a small cooldown
before they can move again.

When dashing onto a bed, the player could be sent flying a short
distance.  A yellow bag could serve as a parachute.

\section{Enemies/Furniture}

\subsection{Types}

\begin{table}[ht!]
  \centering
  \begin{tabular}{lllll}
    \toprule
    \textbf{Name} & \textbf{Type} & \textbf{Movement} & \textbf{Attack}
                  & \textbf{Death}
     \\
    \midrule
    Kallax & shelf & rolls sideways & quartering & falls into pieces? \\
    Lack & small table & gallop & ramming & ? \\
    ? & lamp & small jumps & explodes & dies when attacking \\
    ? & spotlight & small jumps & laser beam & ? \\
    ? & carpet & slides across floor & makes you trip over it & 
        only with fire\tablefootnote{%
          This means you'll have a fast-moving, burning carpet sliding
          around, which may ignite other furniture or yourself.
        } \\
    ? & cutlery & flying swarm & pokes you & melee? better just run
    away? \\
    ? & couch & slow tank & eats you & dead couch can be used as cover\\
    Markus & office chair & very fast & spins quickly & ? \\
    \bottomrule
  \end{tabular}
  \caption{Enemy Furniture}
  \label{tab:furniture}
\end{table}

\subsection{Behavior}

\section{Items}

\subsection{Basic Resources}

\begin{itemize}
  \item \ikea Pencils.
    Act as ammunition and currency.
  \item Food: Restores health, perhaps gives status effects?
    Types: Köttbullar, Hot Dogs, Elk Noodles, Free Refill Soda Cups,
    Kanelbullar, Almond Cake.
  \item \ikea Family Membership Cards.
  \item Yellow Bags
  \item Dolly (Sackkarre)
  \item Hex Key
\end{itemize}

\subsection{Weapons}

\subsubsection{Melee}

\ikea pencils probably aren't very effective.
Perhaps some other \ikea item, or a sword crafted from pencils?

\subsubsection{Crossbow}

A simple crossbow built with \ikea pencils that shoots \ikea pencils.
There's two possible types of upgrades:
The crossbow itself can be upgraded (e.g. made to shoot faster or more
accurate), and an additional item can be attached to change the way it
works.

\begin{itemize}
  \item Laser Pointer/Spotlight: Does Lamp-type damage.
  \item Mirror: Reflects light, can be used to defeat furniture with
    lamps (or lamps themselves).
  \item Pencil sharpener: Increases damage dealt, projectile speed.
\end{itemize}

\subsection{Obtaining Items}

There may be a crafting system (with the cafeterias being the crafting
stations) and/or shops (also cafeteria).

\section{Game Modes}

\begin{itemize}
  \item Main: Clear all layers (\smaland, one of each type in random
    order, warehouse).
  \item Arcade: Special showroom with waves of enemies.
\end{itemize}

\section{Artstyle}

Pixel art,
isometric perspective.

\section{UI}

\subsection{Main Menu}

\ikea entrance, player w/ parents may enter parts of store to choose
game mode.

%-----------------------------------------------------------------------
%\vfill
%\noindent
%\rule{\textwidth}{0.4pt}
%
%\bibliographystyle{natdin}
%\bibliography{refs} % Entries are in the "refs.bib" file
%-----------------------------------------------------------------------
\end{document}
